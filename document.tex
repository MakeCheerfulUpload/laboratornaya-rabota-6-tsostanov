\documentclass[oneside, twocolumn, final, 12pt] {extreport}
% "Русификация" документа
\usepackage[utf8]{inputenc}
\usepackage[english,russian]{babel}
% Красная строка для первого параграфа
\usepackage{indentfirst}
% Добавляет возможность копирования текста из PDF файла
% Без этой строчки в буфер обмены попадает текст в кодировке
% отличной от utf-8 
\usepackage{cmap}

\usepackage{graphicx}
\usepackage{wrapfig}

\begin{document}
	
	\chapter{Расы}
	\section{Гномы}
	\begin{wrapfigure}{r}{0.3\textwidth}
		\includegraphics[width=0.3\textwidth]{E}
		\caption{Гном}
		\label{fig:image}
	\end{wrapfigure}
	$Гномы известны за своё мастерство в искусстве магии. Их способности противостоять магическим воздействиям и неумной тягой к знаниям, а также тонкой обработкой камней и металлов, в которой они могут сравниться с эльфами. Гномы создают уникальные вещи или передают знания народам и расам.$
	
	\subsection{Индивидуальность}
	
	Гномы очень общительны и часто попадаются среди различных, даже очень разнообразных рас в виде архивариусов, собирателей знаний. Они очень общительны, хотя и не любят когда им навязываются. Также они считают, что ум существа зависит от того сколько букв у них в имени. Они презирают тех кто берет в руки оружие, считая что искусство магии и знание гораздо выше. Гномы добры и считают что знания помогут им соединить разрозненные племена. Бывает что гномы пребывают в роли дипломатов и послов. Между своими родственниками дварфами и цвергами они служат прослойкой мирящей их и помогающей работать с волшебными вещами.