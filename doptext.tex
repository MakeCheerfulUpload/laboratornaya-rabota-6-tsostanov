\documentclass[a4paper, 10pt]{article}

\usepackage[russian]{babel}
\usepackage[T2A]{fontenc}
\usepackage[utf8]{inputenc}
\usepackage{multicol}
\usepackage{adjustbox}
\usepackage{amsmath}
\usepackage{tikz}
\usepackage{amssymb}


\usepackage{geometry}
\geometry{top=10mm}
\geometry{bottom=5mm}
\geometry{left=10mm}
\geometry{right=10mm}

\usepackage{graphicx}
\graphicspath{ {./images/} }
\usepackage{wrapfig}

\newcommand{\display}[1]{\hbox to 0.32\textwidth
	{\hfil \small #1 \hfil}}

\newcommand{\liner}[4]{\begin{tikzpicture}\draw[ultra thick, #1, #2] (0,0) -- (1.4,0);\node[above left] at (0.3,0) {\scriptsize #3};\node[above right] at (1.1,0) {\scriptsize #4};\end{tikzpicture}}
\newcommand{\angler}[4]{\begin{tikzpicture}\filldraw[#1] (0,0) -- (0,0.6) arc (90:0:1.7 em);\node[below left] at (0,0) {\scriptsize #2};\node[above] at (0,0.6) {\scriptsize #3};\node[right] at (0.6,0) {\scriptsize #4};\end{tikzpicture}}
\newcommand{\relgna}[4]{\begin{tikzpicture}\filldraw[#1] (0,0) -- (0,0.6) arc (90:180:1.7 em);\node[below right] at (0,0) {\scriptsize #2};\node[above] at (0,0.6) {\scriptsize #3};\node[left] at (-0.6,0) {\scriptsize #4};\end{tikzpicture}}

\newenvironment{ramka}[1]{\begin{flushleft}\begin{tabular}{|p{#1}|}
			\hline}{\\\hline\end{tabular}\end{flushleft}\smallskip}

\newcommand{\RomanNumeralCaps}[1]
{\MakeUppercase{\romannumeral #1}}

\setlength{\parskip}{0pt}

\renewcommand{\a}{\alpha}

\usepackage{fancyhdr}
\pagestyle{empty}

\begin{document}
    \begin{minipage}{0.3\textwidth}
		\begin{tikzpicture}[scale=1.4, line width=2pt]
			\filldraw[orange] (1.7,8) -- (1.7,8.5) arc (90:0:1.4 em);
			\filldraw[yellow] (1.7,8) -- (1.7,8.5) arc (90:180:1.4 em);
			\draw[blue] (1.7,8) -- (1.7,11);
			\draw[dashed, orange] (0,8) -- (1.7,11);
			\draw[dashed, black] (0,8) -- (1.7,8);
			\draw[orange] (3.4,8) -- (1.7,11);
			\draw[black] (1.7,8) -- (3.4,8);
			\node[below left] at (0,8) {D};
			\node[below right] at (3.4,8) {C};
			\node[below] at (1.7,8) {A};
			\node[above] at (1.7,11) {B};
			\draw[white] (0,0) -- (1.7,0);
		\end{tikzpicture}
	\end{minipage}
	\hfill
	\begin{minipage}{0.61\textwidth}
		{\hfil \Large $74$\hspace*{1.6 em} КНИГА \RomanNumeralCaps{1} ПРЕДЛ. \RomanNumeralCaps{48}. ТЕОРЕМА}
		\vspace*{1.6 em}
		\begin{wrapfigure}{l}{0.18\textwidth}
			\includegraphics[width=0.18\textwidth]{E}
		\end{wrapfigure}
		\phantom{Я ненавижу Latex}\\
		\textit{\large сли в треугольнике квадрат одной стороны }
		\liner{}{orange}{B}{C}
		\textit{\large равен сумме квадратов двух других сторон}
		\liner{}{blue}{A}{B}
		\textit{\large и}
		\liner{}{black}{A}{C}
		\textit{\large ,то угол\\}
		\angler{orange}{A}{B}{C}
		\raisebox{20pt}{\itshape \large\\ ,заключенный между этими двумя}
		\textit{\large сторонами прямой.}
		\bigskip
		\begin{center}
			\rule{0pt}{1 em}
			\Large Проведем
			\liner{dashed}{black}{A}{D}
			$\perp$
			\liner{}{blue}{A}{B}
			\\ \vspace{3mm}
			и $=$
			\liner{}{black}{A}{C}
			(пр. $\textrm{\RomanNumeralCaps{1}}._{\textrm{\RomanNumeralCaps{2}}},\textrm{\RomanNumeralCaps{1}}._3$)
			\\ \vspace{3mm}
			также проведем
			\liner{dashed}{orange}{B}{D}
			.\\ \vspace{3mm}
			Поскольку
			\liner{dashed}{black}{A}{D}
			$=$
			\liner{}{black}{A}{C}
			(постр.)\\ \vspace{3mm}
			\liner{dashed}{black}{A}{D}
			\raisebox{8pt}{\small 2} $=$
			\liner{}{black}{A}{C}
			\raisebox{8pt}{\small 2} ;\\ \vspace{3mm}
			$\therefore$
			\liner{dashed}{black}{A}{D}
			\raisebox{8pt}{\small 2} $+$
			\liner{}{blue}{A}{B}
			\raisebox{8pt}{\small 2} $=$
			\liner{}{black}{A}{C}
			\raisebox{8pt}{\small 2} $+$
			\liner{}{blue}{A}{B}
			\raisebox{8pt}{\small 2} ;\\ \vspace{3mm}
			но
			\liner{dashed}{black}{A}{D}
			\raisebox{8pt}{\small 2} $+$
			\liner{}{blue}{A}{B}
			\raisebox{8pt}{\small 2} $=$
			\liner{dashed}{orange}{B}{D}
			\raisebox{8pt}{\small 2}
			(пр. $\textrm{\RomanNumeralCaps{1}}._{47}$),\\ \vspace{3mm}
			и
			\liner{}{black}{A}{C}
			\raisebox{8pt}{\small 2} $+$
			\liner{}{blue}{A}{B}
			\raisebox{8pt}{\small 2} $=$
			\liner{}{orange}{B}{C}
			\raisebox{8pt}{\small 2}
			(гип.)\\ \vspace{3mm}
			$\therefore$
			\liner{dashed}{orange}{B}{D}
			\raisebox{8pt}{\small 2} $=$
			\liner{}{orange}{B}{C}
			\raisebox{8pt}{\small 2},\\ \vspace{3mm}
			$\therefore$
			\liner{dashed}{orange}{B}{D} $=$
			\liner{}{orange}{B}{C};\\ \vspace{3mm}
			\raisebox{15 pt} { и $\therefore$}
			\normalsize
			\relgna{yellow}{A}{B}{D} \raisebox{15 pt}{$=$}
			\angler{orange}{A}{B}{C} \raisebox{15 pt}{\Large (пр. \RomanNumeralCaps{1}.8),}\\ \vspace{3mm}
			\raisebox{15 pt}{\Large следовательно}
			\angler{orange}{A}{B}{C}
			\raisebox{15 pt}{\Large прямой угол.}\\ \vspace{3mm}
		\end{center}
		\Large \mbox{}\hfill ч. т. д.
	\end{minipage}


\end{document}